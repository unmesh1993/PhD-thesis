%%% Thesis Introduction --------------------------------------------------
\chapter{Introduction}\label{introduction}
\graphicspath{{Introduction/Figures/}}

Quantum Mechanics (QM) has successfully provided a framework to accurately describe the properties of matter based on a probabilistic approach. Even though innumerous developments have been made in the field of QM, many new areas are emerging and expanding. In this thesis, I have made an effort to understand  and contribute in the area of ``Nuclear Quantum Effect" (NQE).

\section {What are NQEs?}
In 1925, Erwin Schr\"{o}dinger postulated the famous Schr\"{o}dinger equation (SE) to describe the state of a quantum mechanical system. The SE is the quantum mechanical analogue of Newton's second law of motion and  an exact solution is capable of describing the system and its properties accurately. However, the solution becomes intractable and unsolvable for realistic systems. Therefore, multiple approximations have been introduced over the years to tackle the quantum mechanical problem. The most common approximation is to describe the nuclei as classical point like objects. This classical description helped in explaining most of the physical observations like motion of atoms in fluids, rates of chemical and biological processes and many other macroscopic properties that depend on the state of the nuclei. With the advancement of experimental techniques, many noticeable observations primarily at cryogenic temperatures like phase transition\cite{horiuchi2008organic}, kinetic isotope effects\cite{truong2021large}(rate of lighter isotope / rate of heavier isotope), enhanced thermal rates\cite{zuev2003carbon} (in contrast to the expected value from the empirical Arrhenius law) could not be explained. The theoretical developments that describe the nuclei quantum mechanically proved beneficial in addressing experimental observations predominantly linked to the state of the nuclei.  These real world manifestations of the nuclei are known as ``Nuclear Quantum Effect (NQE)". 

\noindent The major contributors to the NQEs are Zero Point Energy effects (ZPE) and quantum tunnelling. Although either effect has synergistic influence on the observed properties, under certain conditions the individual effects become pronounced. For example, the ability of He$^4$ to be in the liquid state at 0 K\cite{london1936condensed} ($<$ 25 atm) is primarily due to ZPE whereas point mutation in DNA is largely dependent on proton tunnelling\cite{slocombe2022open,rein1964proton} between base pairs. 
\noindent NQE are generally detected from isotopic substitution and Inelastic Neutron Scattering (INS) studies. In the former method, the light nuclei is replaced by a heavier isotope such that any variation in the observed properties of the two iso-electronic nuclei is due to the quantum nature of the nuclei itself. Typically, the kinetic isotope effects ($k_{light}/k_{heavy}$, where $k_{light/heavy}$ is the reaction rate consisting of light/heavy isotope) are compared where a larger value corresponds to an enhanced quantum effect of the nuclei. The latter, INS method, is based on the inelastic scattering of neutrons with the atomic nuclei. As a result, the amount of energy lost by the neutrons gets transferred to the quantized lattice vibrations of the nuclei. The lattice vibrations corresponding to the quantum tunnelling also gets detected in the process. For the specific case of hydrogen nuclei, NQEs are detected using deep inelastic neutron scattering (DINS), Scanning Tunnelling Microscopy\cite{cahlik2021significance} (STM) and Helium spin echo\cite{jardine2010determination} (Hese) experiments to name a few. 

NQE are exhibited in diverse domains such as biological systems\cite{riaz2013review,kohen1999hydrogen,nagel200921st,pu2006multidimensional}, astrochemistry\cite{hama2013surface,jorgensen2020astrochemistry}, important organic reactions\cite{meisner2016atom,castro2020heavy}, catalysis\cite{klinman2006role,knapp2002environmentally} and chemical physics\cite{leger2019observation,cazorla2017simulation}. Some of the notable examples of NQE include ring expansion of carbene\cite{zuev2003carbon} at 8 K, competitive tunnelling of nitrogen vs carbon in p-aminophenylnitrene\cite{leger2019observation},  new tunnelling state of water in Beryl\cite{kolesnikov2016quantum}, mass-independent isotope effect in ozone\cite{thiemens2001mass,thiemens1999mass}, inverse kinetic isotope effect of hydrogen diffusion in zeolites\cite{gao2019quantum}, stabilization of Ketosteroid isomerase\cite{wang2014quantum}, quantum effects in brain\cite{adams2020quantum} and proton-coupled energy transfer in molecular triad\cite{pettersson2022proton} to name a few. 




%\noident The exact modelling of NQE is not straightforward and becomes intractable to implement in computer codes. However, there has been continuous development to find viable options which includes methods based on semi classical dynamics\cite{heller1975time,wang1998semiclassical}, modification to the existing transition state theory\cite{fernandez2007variational}, the dispersed polaron method\cite{warshel1986simulation}, imaginary time path integrals\cite{marx1994ab,feynman2010quantum}, the nuclear-electron orbital method\cite{multicomponentshs,webb2002multiconfigurational}, non-Born-Oppenheimer and multi component DFT\cite{capitani1982non,kreibich2001multicomponent}. Amongst these many methods,  the path integral molecular dynamics\cite{marx1994ab} (PIMD) method (based on the work by Richard Feynman\cite{feynman2010quantum}) has been proved to be very effective in handling realistic systems. The method uses classical dynamics to sample the quantum partition function by transforming the quantum nucleus into a ring polymer of ``$N$" pseudo-classical nuclei/beads such that in the limit ``$N$" tends to $\infty$ the mapping becomes exact. The above method ensures the revelation of any zero point and tunnelling effects of the nuclei. The above method is only ``$N$" times more expensive than the classical counterpart and scales linearly with the number beads ($N$).

\section{NQEs in hydrogen-bonded (HB) systems}
\noindent The exhibitions of NQE depend inversely on the mass of the nuclei. Hydrogen atom has the lightest nucleus in the periodic table and therefore, has the maximal quantum nature. Hydrogen atom constitutes around 90 \% of the universe by weight and exists as hydrogen molecule, hydrocarbons, nuclear bases and inorganic compounds like water. Amongst the diversity, hydrogen atoms in HB species are found to experience extreme quantum delocalization\cite{markland2018nuclear,ceriotti2016nuclear}.  Water, the most common of all HB systems, owes its pH value of 7 \cite{ceriotti2016nuclear} and the specific heat capacity of 4.2 J (g°C)$^{-1}$ \cite{vega2010heat} to the room temperature manifestations of NQE present in the hydrogen bonds. In biological systems, the anomalously faster rates of reaction in enzyme catalysis could only be explained through the quantum tunnelling of protons\cite{sutcliffe2000enzymology,riaz2013review}. Similarly, other HB systems such as perovskites\cite{pena2001chemical,feng2018proton,chen2017kinetic}, metal-organic frameworks\cite{meng2017proton,teufel2013mfu}, hydrogen-bonded molecular solids\cite{wikfeldt2014communication,ivanov2015quantum} and nuclear bases\cite{kim2021quantum} also show the mammoth presence of NQE. In this thesis, I have investigated NQE in two specific systems, (i) multiple hydrogen transfers in hydrogen bonded molecular solids and (ii) structural and electronic properties of water at the metal-water interface.


\subsection{Multiple hydrogen transfers in HB molecular solids}

\noindent Every hydrogen bond can be considered as an incipient hydrogen transfer reaction and the distance between the heavy atoms determine the nature of transfer. A very short separation ($<$ 2.3 \AA{}) corresponds to a barrierless transfer like in [F-H-F]$^{1-}$ ion\cite{dereka2021crossover} whereas a large separation ($>$ 3.5 \AA{}) corresponds to the localised state like in weak ``improper" C-H$\cdots$O hydrogen bonds\cite{desiraju2001weak}. In the intermediate range like in water, the transfer is associated with an activation barrier and the magnitude of the barrier can be tuned by varying the separation. The transfer of hydrogen atoms along the hydrogen bonds lead to macroscopic manifestations like electrical conductivity in acidic water, ionic liquids and biological channels where the hydrogen ion diffuses via Grotthuss mechanism. Excited state proton transfer reactions\cite{zhou2018unraveling} are observed in many biological systems, such as DNA, photosystem II and green fluorescence protein. Moreover, proton transfer reaction can also be used as a mass spectroscopic technique\cite{blake2009proton}. 

\noindent In hydrogen bonded molecular solids, the heavy atom separation typically falls in the intermediate range\cite{gilli2000towards,steiner2002hydrogen} and the proton transfer is associated with an activation barrier. Therefore, the transferring hydrogen experiences a double-well potential such that the minima of each well represent the equilibrium distance of the covalently bonded hydrogen from the donor heavy atom. At low temperatures (near 0 K), the hydrogen atom is localised in one of the two wells and stays indefinitely bonded to the donor heavy atom. This phase of the hydrogen bond is known as the ``Ordered phase". An increase in temperature increases the vibrational energy of the hydrogen atom and it starts to shuttle between the two wells. This dynamic state of hydrogen is known as the ``Disordered phase". The transition from the ordered to the disordered state in HB solids result in interesting properties like anti-ferroelectric to paraelectric phase transition in squaric acid, ferroelectric to paraelectric phase transition in croconic acid \cite{horiuchi2008organic} and KDP\cite{engel2018spatially,srinivasan2011isotope}. Moreover, hydrogen transfers in the disordered state of the solids also has important implications, such as the interconversion of the two crystalline forms of a molecular magnet\cite{armentano2005intermolecular} and 3D structure determination through proton-proton dynamics\cite{lange2003analysis}. Different experimental techniques like Nuclear Magnetic Resonance, inelastic neutron scattering, Raman spectroscopy and ultrafast transient spectroscopy are used to describe the nature of hydrogen transfers. However, most hydrogen transfers occur at a time scale smaller than the resolution of these experiments and consequently, the exact description gets masked under an averaged value. Therefore, it is necessary to have a theoretical perspective to obtain the finer details which are otherwise not available through experiments. Hence, many theoretical studies\cite{wikfeldt2014communication,litman2020temperature,sttoceek2022importance,fallacara2021thermal,rossi2016anharmonic} have been conducted to understand NQE in HB  molecular solids. In Chapter-3 of this thesis, we investigate proton transfer processes in the molecular crystals of Terephthalic Acid .

\subsection{Structural and electrochemical properties of water at the metal/water interface}

\noindent Water, the most common HB system, is central to the development of clean energy. In electrochemical cells, water as an electrolyte promotes the inter-conversion of chemical and electrical energy. Specifically, Hydrogen Evolution Reaction (HER), where water is converted to hydrogen fuel, is facilitated by electrocatalysts. Pt metal with low overpotential\cite{li2019recent,eftekhari2017electrocatalysts} is till date the most efficient commercial electrocatalyst for HER. Similarly, other noble metals like Ru, Ir and Pd have shown promising results with respect to the overpotential\cite{li2019recent,sarkar2018overview}. The cheaper alternatives based on alloys, transition metal compounds and carbonaceous nanomaterials are also being investigated\cite{eftekhari2017electrocatalysts}. Among them, the transition metal compounds like MoS$_2$\cite{jaramillo2007identification} and WC\cite{fan2015wc} have shown great promise and possess Pt like activity for HER. The HER performances of a collection of transition-metal based electrocatalysts are listed in the review\cite{eftekhari2017electrocatalysts} by Ali Eftekhari. 

\noindent The rates of the electrochemical processes in  HER depend on the  metal and their interaction with water. The macroscopic properties like the electrode potential and conversion ratio (of water to hydrogen molecule) are captured using experimental techniques. Modifications or improvement to these macroscopic properties can only be done when the microscopic properties of the interface are well understood.  The structure of water, presence of charged species, effect of pre-adsorbed atoms on the metal surface and the charge transfer between the metal and water are some of the microscopic properties that need to be thoroughly investigated. Surface scientists have used experimental techniques like atomic force microscopy\cite{tian2022visualizing} (AFM), Scanning tunnelling Microscopy\cite{gewirth1997electrochemical} (STM) and others\cite{magnussen2019toward} under ultra high vacuum conditions to understand the microscopic structure of the interface water. Though these studies have provided a great deal of understanding towards the development of electrocatalysts, the experimental conditions and the interpretations are often far from exact. Hence, a theoretical perspective is necessary to support the experimental findings. Since the water-water and water-metal interactions are comparable and compete with each other, it is important to accurately investigate the interface using quantum mechanical simulations. Over the years, the interfaces have been investigated using \textit{ab initio} studies where only the electrons were treated quantum mechanically. Although most of the observations could be explained, the presence of light hydrogen nuclei in water gives rise to the question of whether NQEs are important for these systems. The huge computational requirement in modelling NQEs has restricted the number of studies to a handful\cite{lan2020ionization,tian2022visualizing,yan2020nuclear}. In chapter-4 of this thesis, we investigated NQEs at the Pt(111)/water interface using a relatively inexpensive method that involves integration of QMMM with the path integral molecular dynamics simulation. 

\section{Computational Modelling of NQEs}
Quantum mechanical solution of realistic systems are not exactly solvable. Hence, approximations are needed. The classical treatment of the nuclei is one of the important approximations adapted to reduce the computational burden. This approximation does not hold good for systems with light nuclei or/and at low temperatures. Therefore, the classical description of the nuclei in hydrogen bonded systems suffer from discrepancies. However, there has been continuous development to find viable options to incorporate NQE. Some of these developments though approximate include methods based on semi classical dynamics\cite{heller1975time,wang1998semiclassical}, modification to the existing transition state theory\cite{fernandez2007variational}, the dispersed polaron method\cite{warshel1986simulation}, imaginary time path integrals\cite{marx1994ab,feynman2010quantum}, the nuclear-electron orbital method\cite{multicomponentshs,webb2002multiconfigurational}, non-Born-Oppenheimer and multi component DFT\cite{capitani1982non,kreibich2001multicomponent}. Amongst the many methods,  the path integral molecular dynamics\cite{marx1994ab} (PIMD) and the path integral monte carlo\cite{herman1982path} (PIMC) method (based on the work by Richard Feynman\cite{feynman2010quantum}) have proved to be very effective in handling realistic systems. The PIMD method uses classical dynamics to sample the quantum partition function by transforming the quantum nucleus into a ring polymer of ``$N$" pseudo-classical nuclei/beads such that in the limit $N \rightarrow \infty$ the mapping becomes exact. The above method ensures the revelation of any zero point and tunnelling effects of the nuclei. The PIMD method is ``$N$" times more expensive than the classical counterpart and scales almost linearly with the number beads ($N$).

\section{Outline of the thesis}

The thesis has been organised into the following chapters:
\\

\noindent \textbf{Chapter-2} presents a brief description of the theoretical and computational methods used to address the scientific questions raised in Chapter-3 and Chapter-4. This chapter  discusses the quantum  mechanical treatment of the electrons, termed ``electron problem", and the nuclei, termed ``nuclear problem" under the Born-Oppenheimer approximation. The ``electron problem" is solved using Density Functional Theory (DFT) in Chapter-3 whereas in Chapter-4, a hybrid approach of DFT and classical force-fields (FF) that goes by the name Quantum Mechanics Molecular mechanics (QMMM) is used. The ``nuclear problem'' is solved using the statistical approach of molecular dynamics based on imaginary time path integrals. 

\noindent \textbf{Chapter-3} investigates NQE in multiple hydrogen transfers for a prototypical system of Terepththalic Acid (TPA). \textit{Ab initio} Path Integral simulations coupled with Generalized Langevin Equation based thermostat (PIGLET) is used to sample the potential energy surface. The simulations show that the nature of hydrogen transfers changes from quantum tunnelling at low temperatures ($<$ 70 K) to barrier hopping at room temperature. 

\noindent \textbf{Chapter-4} reports the manifestations of NQEs at the electrochemical Pt(111)/water interface. Using a novel implementation, we have integrated QMMM with the PIGLET simulation to facilitate cheaper modelling of the metal/water interface. This method preserves the complete (nuclei + electrons) quantum mechanical description of the interface and observes that the NQEs causes spontaneous dissociation of chemisorbed water. The microscopic structure modifies the work function of the water covered metal which NQEs tend to enhance further.

\noindent \textbf{Chapter-5} summarises and provides an outlook on the work undertaken in the preceding two chapters.    
   

%%% ----------------------------------------------------------------------
%%% Local Variables: 
%%% mode: latex
%%% TeX-master: "../thesis"
%%% End: 