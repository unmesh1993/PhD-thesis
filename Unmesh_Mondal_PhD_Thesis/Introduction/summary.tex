%%% Thesis Introduction --------------------------------------------------
\chapter{Summary and Outlook}\label{summary}
\graphicspath{{Introduction/IntroductionFigs/}}

\section{Summary}

Over the past few decades, progress in experimental techniques such as neutron compton scattering, nuclear inelastic scattering, and solid-state NMR, together with advancements in theoretical and computational methodologies, has significantly advanced our understanding of quantum fluctuations in the nuclei (NQEs). These effects are predominantly observed in H-bonded systems due to the light mass of the hydrogen nucleus\cite{markland2018nuclear,li2011quantum,ceriotti2016nuclear}. However, NQEs have also been observed in heavier nuclei like carbon and sulphur\cite{kundu2021quantum,kundu2022influence,zheng2022nuclear}. In this thesis, using path integral methods we have studied the manifestation of NQEs in two very different hydrogen bonded systems namely molecular crystals of Terephthalic acid and liquid water at the electrochemical Pt(111)/water interface. The former study investigates NQEs in strong hydrogen bonds under the collective influence of the crystalline environment. While the latter study probes NQEs in the most common and well-studied hydrogen bonded system, water\cite{ceriotti2012efficient}, there remains a dearth of studies focused at solid-liquid interfaces, particularly in the context of electrochemical interfaces\cite{litman2018decisive,lan2020ionization,tian2022visualizing,yan2020nuclear}. Additionally, for this work, we have coupled QMMM simulation with PIMD that results in significant speed up of the simulations. In the remainder of this chapter, we have briefly summarized the results obtained from these two studies and discussed the various limitations in our approach. 

\subsection{Terephthalic acid molecular crystal}

\noindent The third chapter of this thesis explores the role of NQEs on multiple proton transfer events in a chemically important molecular crystal, Terephthalic Acid (TPA). TPA crystals are stabilized primarily through two antiparallel intermolecular O-H$\cdots$O hydrogen bonds along with the weak improper intermolecular C-H$\cdots$O hydrogen bonds to preserve the crystalline environment. Additionally, van der Waals interaction also plays an important role. Experiments performed at room temperature\cite{meier1986structure} find that the hydrogen atoms along the strong O-H$\cdots$O bonds become dynamic and shuttle from one oxygen to the other oxygen atom. As a result, either oxygen atoms has almost equal probability of forming covalent bonds with the transferring hydrogen atom. In literature, such a state of the crystal is commonly referred to as the completely ``disordered" phase. On decreasing the temperature, experimental observations show a gradual reduction in the shuttling behaviour and at temperatures below 70 K (``order-disorder" transition temperature) the atom ceases to transfer and preferentially binds to one of the two oxygen atoms. This state of the crystal is known as the ``ordered" state.  

\noindent From our BOMD simulations (at 70, 100, 200, 300 K) where the nuclei were treated classically, we observe proton shuttling events only at 300 K, which is in contradiction with experimental results. However, turning on NQEs resulted in the occurrence of proton shuttling events even at 70 K with gradual enhancement as the temperature was increased. The contrast between the two simulation shows that it is NQEs that cause the order-disorder transition at temperatures below 70 K. In accordance with our chemical intuition we find that the two proton transfers along the two anti-parallel hydrogen bonding are predominately concerted but a small proportion of this transfers also lead to charged intermediates. Our analysis related to the spread of the nuclear wavefunction in terms of the radius of gyration of the hydrogen ring polymer suggests that at 70 K the transfer is dominantly through quantum tunneling whereas at room temperature the thermally activated hopping is the preferred route. At intermediate temperatures a combination of both is observed. 

\noindent The path integral investigation primarily focuses on the statistical quantum effects of the nuclei in order to understand the nature of DPT. The findings are subject to some limitations/approximations. The first is the Born-Oppenheimer approximation (BOA) which is applied to systems where the motion of the electrons are much faster than the nuclei. However, in the case of DPT, the confinement/shuttling of the light hydrogen nuclei between the heavy atoms could make the motion of the hydrogen nuclei and its electron comparable\cite{skone2006calculation} leading to non-adiabatic electronic coupling. This becomes more important in the case of nuclear tunnelling\cite{althorpe2016non}. For example, the non-adiabatic coupling manifests as electronic in-gap end states in 2,5-diamino-1,4-benzoquinonediimines molecules deposited on Au(111) surface\cite{cahlik2021significance}, where the electronic and vibrational proton degrees of freedom are strongly coupled. Moreover, the spin density of the hydrogen atom in non-adiabatic cases might become important due to the involvement of hydrogen radical\cite{althorpe2016non}. These possible shortcomings in our study are both important and interesting but are unfortunately outside the scope of this thesis. 

\noindent Another very important limitation of this study is the use of semi-local functional to approximate the exchange-correlation used in the KS-DFT formulation. These semi-local functionals, due to delocalization error, tend to underestimate the proton transfer barriers\cite{bryenton2023delocalization,ivanov2015quantum}. This results in lowering of the order-disorder transition temperature and an increase in proton shuttling. Nevertheless, the semi-local functionals are used due to their low computational cost and underestimates the proton transfer barrier by only a fraction of eV when compared to methods using higher level of theory as is reported in many proton transfer systems such as formic acid\cite{ivanov2015quantum}, ice\cite{drechsel2014quantum} and 2,5-diamino-1,4-benzoquinonediimines\cite{cahlik2021significance}. 

\noindent Additionally, in the context of DPT, it would be interesting to explore the excited state proton transfer which has important applications\cite{joshi2021excited} as fluorescent probes, proton transfer lasers and organic light emitting diodes. Overall, this chapter shows NQEs are not only important at low temperatures but are also significant at room temperature.  

\subsection{The electrochemical interface of Pt(111)/water}

\noindent In this chapter, we explored the presence of NQEs at the Pt/water interface. Pt(111) electrocatalyst is an efficient electrocatalyst for several important reactions namely such as HER, HOR, ORR and CO$_2$-RR. These electrochemical reactions are governed by the electrode potential of the electrocatalyst which directly or indirectly is associated with the microscopic structures of the interface. Although various experiments\cite{magnussen2019toward,tian2022visualizing} have attempted to understand this correlation but the methods are generally complex and operated under extreme conditions. Hence, a theoretical perspective is mandatory to complement the experimental observations and this forms the basis of our investigation. Pt(111)/water interface is modelled by depositing layers of water on one side of the Pt(111) slab whereas the other side of the slab is designed to face the vacuum. Overall, the system mimics the electrode/electrolyte interaction of an electrochemical half cell. However, the computational modelling of the electrocatalyst using \textit{ab initio} methods is very difficult due the large number of atoms needed to model a realistic system. Alternatively, we adopted the QMMM method which categorises the atoms of the system into ``MM" and ``QM" subsystems. In our model, we assign the atoms of bulk water to the ``MM" subsystem where the electronic structure is described by classical force fields. The remaining atoms consisting of Pt atoms and interfacial water molecules fall under the ``QM" subsystem where the electronic structure is described quantum mechanically. This hybrid approach allowed us to design a large system with over a thousand atoms.

\noindent In this chapter, we modelled two electrochmeical scenarios. The first being the ``clean" Pt(111) surface and the second with half a monolayer of hydrogen atoms (0.5 ML H) on the Pt(111) surface. The latter is an important intermediate in the HER/HOR processes. Similar to the analysis done on TPA, we compared the classical (QMMM) and quantum (QMMM+PIGLET) nuclei simulations to understand the room temperature manifestations of NQEs. The most distinctive observation of NQE is the auto-dissociation of chemisorbed water on the Pt(111) surface into chemisorbed hydroxide and solvated proton, thereby making the interfacial water acidic. This effect is primarily due to the incorporation of zero-point effect in the O-H bond and makes the interfacial water layer acidic. The microscopic arrangement at the interface generates an interfacial dipole which alters the ability of the metal to lose/gain electrons thereby affecting the electrode potential of the metal. The interfacial dipole is composed of two individual dipoles: one generated by charge reorganisation due to the metal-water interaction and the other by preferential orientation of the polar covalent (O-H) bonds. The former eases the electron abstraction process whereas the latter has an opposite effect though comparatively much smaller. NQEs are found to enhance the respective behaviour with the hydrogen covered surface more affected than the clean surface.

\noindent Similar to the previous chapter on DPT, the theoretical approach to understand the electrochemical interface suffers from the limitations of BOA and the use of semi-local functionals to approximate the exchange-correlation. The former approximation excludes the non-adiabatic effects arising from electron-phonon coupling in and between metals and adsorbed gaseous species\cite{alducin2017non,dou2020nonadiabatic,kroger2006electron}. Even for low/zero band gap metals, the coupling is less pronounced due to the disparity in timescales between electron and nuclear motion, with the electron relaxing in the femtoseconds timescale and the nuclei in picoseconds. However, in situations where the charge transfer interaction between the adsorbed species and the metal atoms are ultrafast (in femtosecond time scale), non-adiabatic effect of electronic excitations in the metals take place. The comprehension of this phenomenon is achieved by analyzing the vibrational linewidth of the rapid modes exhibited by the adsorbed species, as observed in the case of CO molecule chemisorbed on the Cu(100) surface\cite{ryberg1985vibrational}. Overall, the investigation of non-adiabatic effects require alternate advanced techniques like surface hopping\cite{sholl1998generalized} which unfortunately is beyond the scope of this thesis. 

\noindent On the other hand, the latter approximation is a very common problem. It is well-established that the use of semi-local functionals lead to over structuring in water\cite{gross2022ab}. Gross and co-workers\cite{forster2014dispersion} used revised PBE (RPBE\cite{hammer1999improved}) with D3 dispersion correction to reduce overstructuring in water. Using this revised semilocal functional they could not observe the formation of water bilayer at the interface. This corresponds to the absence of chemisorbed species which tend to dissociate due to NQEs. However, no molecular dynamics studies of interfaces with hybrid functionals and higher order electronic structure methods have been undertaken to completely negate the presence of the bilayer. 

\noindent Another major challenge/omission in this work is the direct influence of ionic species to the work function change of water covered metal electrode. The ionic/charged species generate electric field which interfere with the electrostatic potential generated by the surface metal atoms thereby affecting the electrode potential. This effect becomes prominent at high and low pH. As a future direction, it would be good to have a complete understanding of all the factors that affect the work function of the water covered metal electrode.

\section{Outlook}
\noindent This thesis was able to provide some unique insights on the importance of  NQEs in two important hydrogen bonded categories. However, many such interesting systems both hydrogen bonded and beyond exist and are still unexplored. Additionally, we investigated primarily the static properties of the quantum nuclei which could be extended to include dynamical properties. Coupling of non-adiabatic effects and electronic spin densities with NQEs are interesting possibilities that needs to be explored through path integral and other novel methods. 


