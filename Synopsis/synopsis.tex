\chapter{Synopsis}

\noindent Nuclei, like electrons, are quantum mechanical objects. However, their heavy mass usually results in a short de Broglie wavelength. Hence, the nuclei are usually highly localized and behave as classical particles. Nevertheless, for light nuclei and even for some heavy nuclei at low temperatures, the wave-like property becomes dominant in many cases and manifests as unexpected phenomenon. Typically, quantum fluctuations of the nuclei evince through effects like zero-point energy, tunnelling, etc. Hydrogen, the lightest nuclei in the periodic table, shows significant nuclear quantum effect (NQE). Consequently, physical properties/processes of/in H-bonded systems are affected by NQEs. For example, accurate prediction of heat capacity of water, hydrogen tunnelling affecting the reaction rates in enzyme catalysis, isotopic substitution enhances the antiferroelectric to paraelectric phase transition temperature in hydrogen-bonded ferroelectrics, and concerted tunnelling in Ih phase of ice are some of the manifestations of NQE in hydrogen-bonded system. In this thesis, using path integral molecular dynamics (PIMD) simulations, I have studied the role of NQEs in two systems: (a) a molecular crystal, terephthalic acid (TPA) and (b) an electrochemical metal/water interface, Pt(111)/water.
\\

\noindent The thesis is organized into five chapters and an appendix. 
\\

\noindent The first chapter (\textbf{Chapter-1}) contains a brief description of the nuclear quantum effects (NQEs) and its manifestations in H-bonded systems. Specifically, I have discussed the role of NQEs in proton transfer processes and the structural and electrochemical properties at the metal/water interface. We also review some of the computational modelling approaches undertaken to incorporate NQEs. At the end, an outline of the thesis is also provided. 

\noindent In the second chapter (\textbf{Chapter-2}) a short description of the theoretical and computational methods used to address the scientific questions raised in the upcoming chapters (\textbf{Chapter-3} and \textbf{Chapter-4}) are presented. This chapter discusses the quantum  mechanical treatment of the electrons, termed ``electron problem", and the nuclei, termed ``nuclear problem" under the Born-Oppenheimer approximation. The ``electron problem" is solved using Density Functional Theory (DFT) in \textbf{Chapter-3} whereas in \textbf{Chapter-4}, a hybrid approach of DFT and classical force-fields (FF) coined Quantum Mechanics Molecular mechanics (QMMM) is used. The ``nuclear problem'' is solved using the statistical approach of molecular dynamics based on imaginary time path integrals. 

\noindent After providing a brief insight into the theoretical foundations of NQEs and path integral methods, I present my PhD research in the third (\textbf{Chapter-3}) and the fourth chapters (\textbf{Chapter-4}). The third chapter (\textbf{Chapter-3}) investigates the temperature dependence of double proton transfer in the molecular crystals of terephthalic acid (TPA). Double proton transfers (DPT) are important for several physical processes, both in molecules and in the condensed phase. While these have been widely studied in biological systems, their study in crystalline environments is rare. In this work, using Path Integral Molecular Dynamics simulations we have studied temperature dependent DPT in TPA. The DPT in TPA results in an order-disorder phase transition. In accordance with experimental reports, we find evidence for this double proton transfer induced order-to-disorder transition that is sensitive to the inclusion of nuclear quantum effects. Our simulations show that the double proton transfer is a concerted process over a wide range of temperatures. At the onset of the transition at low temperatures, DPT occurs through a tunnelling mechanism while at room temperature, activated hopping takes over.

\noindent In the fourth chapter (\textbf{Chapter-4}), room temperature NQEs are studied at the electrochemical Pt(111)/water interface. The metal/water interface models the half cell of an electrochemical reaction and is dependent on the microscopic structure of the interface. It is understood from experiments that the electrode potential of this half cell is related to the work function of the water covered metal surface. This necessitates a complete quantum mechanical description of the interface, which requires prohibitive computational resource. Therefore, using our novel implementation in the electronic structure package of CP2K, we have integrated the Quantum Mechanics Molecular Mechanics method with the state of the art PIGLET simulation. This coupling allowed us to study the role of NQEs on the structural and electrochemical properties at the clean and 0.5 ML H-covered Pt(111)/water interfaces. It is observed that the clean interface has a bilayer which contains chemisorbed water molecules. NQEs spontaneously dissociates the chemisorbed water molecules into hydroxides and protons where the former bind to the Pt surface and the latter gets solvated as Zundel cations. The changes in the microscopic structures of the interface generate an interfacial dipole that alter the work function of the metal. We find that the presence of the water layer reduces the work function and NQEs tend to enhance this reduction further. 

\noindent The fifth and final chapter (\textbf{Chapter-5}) is the ``summary and outlook" where the results obtained from this thesis are summarised, their limitations discussed and possible future directions are envisaged. The \textbf{Appendix} contains the supporting information of the chapters along with two additional works that were undertaken during the course of my PhD but are not included in the main thesis.




